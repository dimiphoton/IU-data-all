\chapter{Hypothesis Test}

\section*{$\xi$ values}

\begin{itemize}
    \item $\xi_{11} = 912$
    \item $\xi_{12} = 36.6$
    \item $\xi_{13} = 2$
    \item $\xi_{14} = [879, 842, 954, 842, 885, 918, 989, 768, 867, 1022]$
\end{itemize}


\section*{Problem statement}

Statistical hypothesis testing is needed to determine whether a new hammer production system yields higher weights than the current system.
The analysis steps include setting up the hypotheses, performing the test, and making a decision based on the results.

\section{Modeling of the hammer weights}

The weights of hammers produced in the factory can be modeled using a normal distribution based on the observed long-term data:
\begin{equation}
W \sim \mathcal{N}(\mu, \sigma^2)
\end{equation}
where:
\begin{itemize}
    \item $\mu = \xi_{11}$ is the mean weight of the hammers.
    \item $\sigma = \xi_{12}$ is the standard deviation of the weights.
\end{itemize}

\subsection*{Assumptions}
The following assumptions are made to ensure that this model holds:
\begin{itemize}
    \item The weights of the hammers are independent .
    \item mean and standard deviation are not varying in time.
\end{itemize}


\section{Hypothesis Testing}

\subsection{Hypotheses definition}

\subsubsection*{Null Hypothesis $H_{0}$}

This is the default assumption or the status quo.  The mean weight of the hammers produced by the new system is the same as the old system: $\mu_{0} = \xi_{11}$.

\subsection*{Alternative Hypothesis ($H_{a}$)}

The alternative hypothesis ($H_{a}$) assumes that the new system produces hammers with a higher mean weight: $\mu_{0} > \xi_{11}$.


\subsection{chosen statistical test and Decision Rule}



\subsubsection*{one-tailed $ t $-test}
The test consists in computing
\begin{equation}
t = \frac{\bar{x} - \mu_0}{s / \sqrt{n}}
\end{equation}
where:
\begin{itemize}
    \item $\bar{x}$ is the sample mean,
    \item $s$ is the sample standard deviation,
    \item $n$ is the sample size,
    \item $\mu_{0} = \xi_{11}$ is the null hypothesis mean.
\end{itemize}

\subsubsection*{Decision rule}

The decision rule is to reject $H_{0}$ if $t>t_{c}$, where $ t_{c} $ is determined from the $ t $-distribution table at a chosen significance level $\alpha$ with $ n-1 = 9 $ degrees of freedom.

To determine whether the new production system yields higher weights, we set up the following hypotheses:
\begin{itemize}
\item $H_{0}$ and $\mu = 912$ : the mean weight remains unchanged
\item $H_{a}$ and $\mu > 912$ :  the mean weight is higher under the new system
\end{itemize}


\subsection{test computation}
The sample weights are:
\begin{equation}
[879, 842, 954, 842, 885, 918, 989, 768, 867, 1022]
\end{equation}

Calculate the sample mean:
\begin{equation}
\bar{x} = \frac{\sum x_i}{n} = \frac{879 + 842 + \cdots + 1022}{10} = 896.6
\end{equation}

Calculate the sample standard deviation:
\begin{equation}
s = \sqrt{\frac{1}{n-1} \sum_{i=1}^n (x_i - \bar{x})^2} \approx 71.5
\end{equation}

Compute the test statistic:
\begin{equation}
t = \frac{\bar{x} - \mu_0}{s / \sqrt{n}} = \frac{896.6 - 912}{71.5 / \sqrt{10}} \approx -0.69
\end{equation}

\subsection*{Decision and Conclusion}
for $\alpha = 0.05$ and $df = 9$, $t_{c} \approx 1.8$. Since $t = -0.69 < t_{c}$, the rule is to not reject $H_{0}$. (equivalent formulation: the $p$ value is $0.54>0.05$.

\textbf{Conclusion}: There is insufficient evidence to suggest that the new system produces hammers with higher weights.
