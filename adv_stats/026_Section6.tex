\chapter{Assignment 6: Bayesian estimate}

\section*{$\xi$ values}

\begin{itemize}
\item $\xi_{17}=15$
\item $\xi_{18}=49$
\item $\xi_{19}=46.65$
\end{itemize}

\section{Problem statement}

We want to make a bayesian modelling of a gamma distribution whose scale parameter is itself a random variable following a gamma distribution.



\section*{Part 1: Posterior Distribution of $\theta$}

The likelihood function for $X$ given $\theta$ is:
\begin{equation}
f(x \mid \theta) = \frac{\beta^\alpha \cdot x^{\alpha-1} e^{-\beta x}}{\Gamma(\alpha)}
\end{equation}
where $\beta = \frac{1}{\theta}$
Substituting $\beta = \frac{1}{\theta}$, we get:
\begin{equation}
f(x \mid \theta) = \frac{\left(\frac{1}{\theta}\right)^3 \cdot x^2 \cdot e^{-\frac{x}{\theta}}}{\Gamma(3)}.
\end{equation}

The prior distribution for $\theta$ is:
\begin{equation}
\pi(\theta) = \frac{\xi_{18}^{\xi_{17}} \cdot \theta^{\xi_{17}-1} \cdot e^{-\xi_{18} \theta}}{\Gamma(\xi_{17})}.
\end{equation}

The posterior distribution $\pi(\theta \mid x)$ is proportional to the product of the likelihood and the prior:
\begin{equation}
\pi(\theta \mid x) \propto f(x \mid \theta) \cdot \pi(\theta).
\end{equation}

Substituting the expressions for the likelihood and prior:
\begin{equation}
\pi(\theta \mid x) \propto \left(\frac{1}{\theta}\right)^3 \cdot x^2 \cdot e^{-\frac{x}{\theta}} \cdot \theta^{\xi_{17}-1} \cdot e^{-\xi_{18} \theta}.
\end{equation}

Simplifying:
\begin{equation}
\pi(\theta \mid x) \propto \theta^{\xi_{17} - 4} \cdot e^{-\xi_{18} \theta - \frac{x}{\theta}}.
\end{equation}

The posterior distribution has the form of a Gamma distribution. For a Gamma distribution, the shape and rate parameters are updated as:
\begin{equation}
\tilde{\alpha} = \xi_{17} + \alpha = \xi_{17} + 3, \quad
\tilde{\beta} = \xi_{18} + x = \xi_{18} + \xi_{19}.
\end{equation}

Thus, the posterior distribution is:
\begin{equation}
\theta \mid x \sim \Gamma(\tilde{\alpha} = \xi_{17} + 3, \tilde{\beta} = \xi_{18} + \xi_{19}).
\end{equation}

\section*{Part 2: Bayes Estimate with Square-Error Loss}

The Bayes estimate under the square-error loss function is the \textbf{mean} of the posterior distribution. 

For a Gamma distribution $ \Gamma(\alpha, \beta) $, the mean is given by:
\begin{equation}
Mean = \frac{\alpha}{\beta}.
\end{equation}

From Part (a), the posterior parameters are:
\begin{equation}
\tilde{\alpha} = \xi_{17} + 3, \quad \tilde{\beta} = \xi_{18} + \xi_{19}.
\end{equation}

Thus, the Bayes estimate is:
\begin{equation}
\hat{\theta}_{mean} = \frac{\tilde{\alpha}}{\tilde{\beta}} = \frac{\xi_{17} + 3}{\xi_{18} + \xi_{19}}=0.188
\end{equation}

\section{Part 3: Bayes Estimate Using the Mode}

The Bayes estimate under the mode of the posterior distribution is the \textbf{mode} of the posterior Gamma distribution. 

For a Gamma distribution $ \Gamma(\alpha, \beta) $, the mode is given by:
\begin{equation}
Mode = \frac{\alpha - 1}{\beta}
\end{equation}

Using the posterior parameters from Part (a):
\begin{equation}
\tilde{\alpha} = \xi_{17} + 3, \quad \tilde{\beta} = \xi_{18} + \xi_{19}
\end{equation}

The mode is:
\begin{equation}
\hat{\theta}_{mode} = \frac{\tilde{\alpha} - 1}{\tilde{\beta}} = \frac{\xi_{17} + 2}{\xi_{18} + \xi_{19}}=0.178
\end{equation}

