\chapter{Transformed Random Variables}

\section*{$\xi$ values}

\begin{itemize}
    \item $\xi_9 = 0$
    \item $\xi_10 = [8,7,6,17,12]$
\end{itemize}



\section{Problem statement}
The total bandwidth to failure $ S $ of a single router follows an exponential distribution with density:
\begin{equation}
f_S(s) = \frac{1}{\theta} \exp\left(-\frac{s}{\theta}\right), \quad s > 0, \, \theta > 0,
\end{equation}
where $ \theta $ is the mean failure bandwidth for a single router.

For a dual-router system, the total bandwidth to failure $ T $ can be expressed as:
\begin{equation}
T = S_{1} + S_{2}
\end{equation}
where $ S_{1} $ and $ S_{2} $ are independent and identically distributed random variables representing the bandwidth totals to failure of each router.

\section{Density Function of $ T $}
Given $ S_{1}$ and $ S_{2}$ both follow an exponential distribution of parameter $\frac{1}{\theta}$, the sum $ T = S_{1} + S_{2} $ follows a \emph{Gamma distribution} with shape parameter $ k = 1+1 $ and rate $ \lambda = \frac{1}{\theta} $. The probability density function of $ T $ with  $ \lambda = \frac{1}{\theta} $ is:
\begin{equation}
f_T(t) = \frac{t \lambda^2 e^{-\lambda t}}{1} = \frac{t}{\theta^2} \exp\left(-\frac{t}{\theta}\right)
\end{equation}

\section{Likelihood Function}
Given an independent sample $ T_1, T_2, \dots, T_n $ of $ T $, the likelihood function for the parameter $ \theta $ is:
\begin{equation}
L(\theta) = \prod_{i=1}^n f_T(T_i) = \prod_{i=1}^n \frac{T_i}{\theta^2} \exp\left(-\frac{T_i}{\theta}\right)
\end{equation}
Simplifying:
\begin{equation}
L(\theta) = \frac{1}{\theta^{2n}} \prod_{i=1}^n T_i \exp\left(-\frac{1}{\theta} \sum_{i=1}^n T_i\right)
\end{equation}

\section{Simplification of the Likelihood Function}
To maximize the likelihood function, we simplify using the log-likelihood:
\begin{equation}
\ell(\theta) = \ln L(\theta) = -2n \ln \theta + \sum_{i=1}^{n} \ln T_{i} - \frac{1}{\theta} \sum_{i=1}^n T_i
\end{equation}
The derivative of $ \ell(\theta) $ with respect to $ \theta $ is:
\begin{equation}
\frac{\partial \ell}{\partial \theta} = -\frac{2n}{\theta} + \frac{1}{\theta^2} \sum_{i=1}^n T_i.
\end{equation}

The model with maximal likelihood \emph{wrt} the parameter $\theta$ has its likelihood derivative equal to zero.

Setting this equal to zero gives:
\begin{equation}
\hat{\theta} = \frac{1}{2n} \sum_{i=1}^n T_i.
\end{equation}

\section{Estimation and Expectation for the Experiment}
Given the sample $ [8, 7, 6, 17, 12] $, the parameter $\hat{\theta}$ for which the model $T$ has the maximal likelihood is:
\begin{equation*}
    \hat{\Theta}=\frac{T}{2} = 5
\end{equation*}

The expected value of $ T $ under MLE is 10.

